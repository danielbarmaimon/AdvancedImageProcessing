%%%%%%%%%%%%%%%%%%%%%%%%%%%%%%%%%%%%%%%%%%%%%%%%%%%%%%%%%%%%%%%%%%%%%%%%%%%%%%%%
%2345678901234567890123456789012345678901234567890123456789012345678901234567890
%        1         2         3         4         5         6         7         8


\documentclass[a4paper, 10pt, conference] {article}        
                                                           

%\IEEEoverridecommandlockouts                              % This command is only
                                                          % needed if you want to
                                                          % use the \thanks command
%\overrideIEEEmargins
% See the \addtolength command later in the file to balance the column lengths
% on the last page of the document



% The following packages can be found on http:\\www.ctan.org
\usepackage{graphicx} % for pdf, bitmapped graphics files
\usepackage{caption}
\usepackage{float}
\usepackage{subcaption}
\usepackage{epsfig} % for postscript graphics files
\usepackage{mathptmx} % assumes new font selection scheme installed
\usepackage{times} % assumes new font selection scheme installed
\usepackage{amsmath} % assumes amsmath package installed
\usepackage{amssymb}  % assumes amsmath package installed
\usepackage[margin=1.5cm]{geometry}
\usepackage{mathtools}
%\usepackage{enumerate}
\usepackage{enumitem}
\usepackage{lipsum}
\begin{document}
\date{}
\title{\LARGE \bf
B31SE2 – Matlab Lab 1
Fourier Theory. Image Analysis in Frequency Domain
}

\author{ \parbox{5 in}{\centering Daniel Barmaimon \\
%         \thanks{*Use the $\backslash$thanks command to put information here}\\
         \ Advanced Image Analysis - Vibot Master Degree\\
         \ Heriot Watt University\\         
}}

\maketitle
%\thispagestyle{empty}
%\pagestyle{empty}


\section{Introduction}
The main purpose of this lab is to understand the filtering over the frequency domain and how could the quality of the image be affected by the compression of the phase and magnitude spectrum.
%\begin{equation}
%g(x,y) = H[f(x,y)]+ n(x,y) 	
%\end{equation} 
%Where \textit{$f(x,y)$} is the original image, \textit{$n(x,y)$} is the model of the noise, and \textit{$H$} is a linear operator that represents the transformation for blurring the image.
%The two kernels that are going to be used during this lab are Gaussian and motion, and could be seen in Fig.\ref{kern}
%\begin{figure}[H]
% 	\centering
% 	\begin{subfigure}{0.49\textwidth} 
% 		\centering						
% 		\includegraphics[scale=0.3]{gaussian/no_noise/kernel_sigma2.PNG}
%		\caption{Gaussian kernel}
%	\end{subfigure}
%	\begin{subfigure}{0.49\textwidth}
%		\centering
%		\includegraphics[scale=0.3]{motion/kernel_alpha30.PNG}
%		\caption{Motion kernel}
%	\end{subfigure}
%	\caption{Kind og kernels for blurring the image}
%	\label{kern}
%\end{figure}
%
%
%As \textit{$H$} is a convolution the blurred image could be expressed as follows. 
%\begin{equation}
%g(x,y) = \iint h(x-u, y-v) \cdot f(u,v) \cdot dudv + n(x,y)
%\end{equation}
%For reasons of computation it recommendable to transform this equation to the frequency domain having as result the following expression.
%\begin{equation}
%G(u,v) = H(u,v) \cdot F(u,v) + N(u,v)
%\end{equation}
%%%
\section{Experiments}
\subsection{Phase and magnitude manipulation}
The image in this example will be split into magnitude and phase, and it will be reconstructed later on from this two parts, allowing to show the inversibility of the Fourier transform in absence of noise. 
 \begin{figure}[H]
 	\centering
 	\includegraphics[width=0.5\textwidth]{reportImages/exp1_original.png} %
 	\caption{Original image}
 	\label{original}
 \end{figure} 
The image could be seen in the color map to allow a better visualization. From the original image the magnitude and the phase could be obtained applying  the Fourier transform for each pixel. This transformation and the magnitude and phase images are calculated as follows.
\begin{eqnarray}
F(u,v)=\sum_{x=0}^{N}\sum_{y=0}^{M}f(x,y)\exp^{-2\pi j\left(\frac{xy}{M}+\frac{yv}{N}\right)}\\
\left|F(u,v)\right|=\sqrt{Re^{2}\left(F(u,v)\right)+Im^{2}\left(F(u,v)\right)}\\
\angle F(u,v) = \arctan\left(\frac{Im\left(F(u,v)\right)}{Re\left(F(u,v)\right)}\right)
\end{eqnarray}
Where \textit{$Im\left(F(u,v)\right)$} represents the imaginary part of the Fourier transform while \textit{$Re\left(F(u,v)\right)$} represents the real part.
The results of getting each one of these images and the posterior reconstruction could be seen in Fig.\ref{exp1}
\begin{figure}[H]
 	\centering
 	\begin{subfigure}{0.32\textwidth} 
 		\centering						
 		\includegraphics[scale=0.5]{reportImages/exp1_magnitude.PNG}
		\caption{Magnitude of the Fourier transform image}
	\end{subfigure}
	\begin{subfigure}{0.32\textwidth}
		\centering
		\includegraphics[scale=0.5]{reportImages/exp1_phase.PNG}
		\caption{Phase of the Fourier transform image}
	\end{subfigure}
	\begin{subfigure}{0.32\textwidth}
		\centering
		\includegraphics[scale=0.5]{reportImages/exp1_reconstructed.PNG}
		\caption{Reconstructed image}
	\end{subfigure}
	\caption{Reconstruction of an image from its magnitude and phase}
	\label{exp1}
\end{figure}
The magnitude is giving a measure of the how much is any frequency in the original image, while the phase is given the position of each of these frequencies in the image. For mathematical reasons the Fourier transform image is always symmetrical respect the origin, so is centered at this position and displayed in a logarithmic scale.\\

In the next example the reconstruction of the image will try to be implemented after making a random phase for each of the pixels. In this way all the space information will be lost and the reconstruction will not be nearly close to the original image.
\begin{figure}[H]
	\centering
	\begin{subfigure}{0.49\textwidth} 
		\centering						
		\includegraphics[scale=0.5]{reportImages/exp1_lena.PNG}
		\caption{Original image: Lena}
	\end{subfigure}
	\begin{subfigure}{0.49\textwidth}
		\centering
		\includegraphics[scale=0.5]{reportImages/exp1_random_phase.PNG}
		\caption{Reconstruction with a random phase}
	\end{subfigure}
	\label{exp1_1}
\end{figure}
If the same experiment is performed over an image with low space correlation, the reconstructed image will have a similar appearance. The gray levels should have the same number of pixels for both of the images.

\begin{figure}[H]
	\centering
	\begin{subfigure}{0.49\textwidth} 
		\centering						
		\includegraphics[scale=0.5]{reportImages/exp1_sar.PNG}
		\caption{Original image: Sar}
	\end{subfigure}
	\begin{subfigure}{0.49\textwidth}
		\centering
		\includegraphics[scale=0.5]{reportImages/exp1_sar_random_phase.PNG}
		\caption{Reconstruction with a random phase}
	\end{subfigure}
	\label{exp1_2}
\end{figure}

In the case that the magnitude were replaced by random values, the spatial information will remain, but not the intensities for each pixel. The reconstruction for the image could be seen in the next picture.

 \begin{figure}[H]
 	\centering
 	\begin{subfigure}{0.49\textwidth} 
 		\centering						
 		\includegraphics[scale=0.5]{reportImages/exp1_lena.PNG}
 		\caption{Original image: Lena}
 	\end{subfigure}
 	\begin{subfigure}{0.49\textwidth}
 		\centering
 		\includegraphics[scale=0.37]{reportImages/exp1_random_mag.PNG}
 		\caption{Reconstruction with a random magnitude}
 	\end{subfigure}
 	\label{exp1_3}
 \end{figure}
 Repeating the experiment for \textit{'Sar.bmp'} is possible to appreciate that no special information is remaining.
 \begin{figure}[H]
 	\centering
 	\begin{subfigure}{0.49\textwidth} 
 		\centering						
 		\includegraphics[scale=0.5]{reportImages/exp1_sar.PNG}
 		\caption{Original image: Lena}
 	\end{subfigure}
 	\begin{subfigure}{0.49\textwidth}
 		\centering
 		\includegraphics[scale=1]{reportImages/exp1_sar_random_mag.PNG}
 		\caption{Reconstruction with a random magnitude}
 	\end{subfigure}
 	\label{exp1_4}
 \end{figure}

\section{Fourier analysis and filtering}
\subsection{Simple Fourier filtering}
In this case it is going to be shown how is possible to perform a low-pass filtering by removing the higher frequencies from a certain image.

\begin{figure}[H]
	\centering
	\begin{subfigure}{0.32\textwidth} 
		\centering						
		\includegraphics[scale=0.5]{reportImages/exp2_lena.PNG}
		\caption{Original image}
	\end{subfigure}
	\begin{subfigure}{0.32\textwidth}
		\centering
		\includegraphics[scale=0.5]{reportImages/exp2_lena_filer_01.PNG}
		\caption{Cutting off the high frequencies}
	\end{subfigure}
	\begin{subfigure}{0.32\textwidth}
		\centering
		\includegraphics[scale=0.5]{reportImages/exp2_lena_filered_01.PNG}
		\caption{Filtered image}
	\end{subfigure}
	\caption{Reconstruction of an image from its magnitude and phase}
	\label{exp2_lowPass}
\end{figure} 
 The problem with this kind of filtering is that the noise, that is in high frequencies, is being removed, but the edges are being removed as well. The use of removing certain frequencies using this filter is great when the noise has a known frequency, then a notch filter could be applied. 
 
\subsection{Developing FFT package}
 In this section  the code to compute the FFT will be developed. The following steps has been followed.
 \begin{enumerate}
 	\item{Multiply the input image by (-1)x+y to center the transform for filtering}. It will allow to concentrate low frequencies in the center of the image
 	\item{Compute the 2-D DFT and  the spectrum} Once in the frequency domain the filtering would be possible.
 	\item{Multiply the resulting (complex) array by a real filter function (in the sense that the the real coefficients multiply both the real and imaginary parts of the transforms).} This is the way of choosing which frequencies will be cut and which ones will remain.
 	\item{Compute the inverse 2-D Fourier transform.} Changing from the frequency domain to the space domain.
 	\item{Multiply the result by (-1)x+y and take the real part} It is necessary to recover the original position of each pixel. 
 \end{enumerate}
 The original image, mask for the filter and result could be seen in Fig.\ref{exp2_FFT}
 \begin{figure}[H]
 	\centering
 	\begin{subfigure}{0.32\textwidth} 
 		\centering						
 		\includegraphics[scale=0.5]{reportImages/exp1_lena.PNG}
 		\caption{Original image}
 	\end{subfigure}
 	\begin{subfigure}{0.32\textwidth}
 		\centering
 		\includegraphics[scale=0.336]{reportImages/mask.PNG}
 		\caption{Cutting off the high frequencies}
 	\end{subfigure}
 	\begin{subfigure}{0.32\textwidth}
 		\centering
 		\includegraphics[scale=0.336]{reportImages/lena_FFT.PNG}
 		\caption{Filtered image}
 	\end{subfigure}
 	\caption{Use of the FFT to filter an image}
 	\label{exp2_FFT}
 \end{figure} 
 
 \subsection{Fourier Spectrum and Average Value}
 There is a relationship between the maximum value in the Fourier domain and the average value of the image in the space domain. The maximum value (the center) in the Fourier domain divided by the total amount of pixels is the average value in the original image.
 \begin{figure}[H]
 	\centering
 	\includegraphics[width=0.3\textwidth]{reportImages/exp2_3_spectrum.png} %
 	\caption{Spectrum of FFT for 'lena.png'}
 	\label{Average}
 \end{figure}
 The average value in the original image is 124.0505, while the one obtained from the center of the spectrum is exactly the same .
 
 \section{Compression and DCT}  
 Last experiment will consist in analyze how precision of the quantization in the spectrum will affect to the results of the image.
 First, let's check how the quality of the reconstructed image is affected when the original image of Lena is quantized in magnitude but not in phase. This will be achieved setting the maximum levels of quantization (512x512) for the phase to the total amount of pixels in the image and the levels of quantization for the phase as the closer integer to the 10\% of the number of pixels of the image.
  \begin{figure}[H]
  	\centering
  	\begin{subfigure}{1\textwidth} 
  		\centering						
  		\includegraphics[scale=0.5]{reportImages/exp3_magnitude01.PNG}
  		\caption{Original image, magnitude, and phase}
  	\end{subfigure}
  	\begin{subfigure}{1\textwidth}
  		\centering
  		\includegraphics[scale=0.5]{reportImages/exp3_magnitude_compressed01.PNG}
  		\caption{Quantized in magnitude and phase}
  	\end{subfigure}
  	\begin{subfigure}{0.32\textwidth}
  		\centering
  		\includegraphics[scale=0.336]{reportImages/exp3_lena_magnitude_compressed01.PNG}
  		\caption{Result of reconstruction}
  	\end{subfigure}
  	\caption{Quantization of the magnitude to 10\% of original size}
  	\label{exp3_mag}
  \end{figure}
  
  If  the process is repeated for different values of the quantization level for the magnitude the results are like in Fig.\ref{exp3_mag_1}
  \begin{figure}[H]
  	\centering
  	\begin{subfigure}{0.32\textwidth} 
  		\centering						
  		\includegraphics[scale=0.33]{reportImages/exp3_mag001.PNG}
  		\caption{1\% of NxM = 2622}
  	\end{subfigure}
  	\begin{subfigure}{0.32\textwidth}
  		\centering
  		\includegraphics[scale=0.33]{reportImages/exp3_mag0001.PNG}
  		\caption{0.1\% of NxM = 262}
  	\end{subfigure}
  	\begin{subfigure}{0.32\textwidth}
  		\centering
  		\includegraphics[scale=0.33]{reportImages/exp3_mag00001.PNG}
  		\caption{0.01\% of NxM = 26}
  	\end{subfigure}
  	\caption{Varying the quantization levels for the magnitude}
  	\label{exp3_mag_1}
  \end{figure}
   
  In an analog way is possible to reduce the amount of levels for the quantization with respect to the phase. It will be shown that a lower amount of levels will be needed to recover the image with an acceptable quality.
  \begin{figure}[H]
  	\centering
  	\begin{subfigure}{0.24\textwidth} 
  		\centering						
  		\includegraphics[scale=0.3]{reportImages/exp3_phase001.PNG}
  		\caption{1\% of NxM = 2622}
  	\end{subfigure}
  	\begin{subfigure}{0.24\textwidth}
  		\centering
  		\includegraphics[scale=0.3]{reportImages/exp3_phase0001.PNG}
  		\caption{0.1\% of NxM = 262}
  	\end{subfigure}
  	\begin{subfigure}{0.24\textwidth}
  		\centering
  		\includegraphics[scale=0.3]{reportImages/exp3_phase00001.PNG}
  		\caption{0.01\% of NxM = 26}
  	\end{subfigure}
  	\begin{subfigure}{0.24\textwidth}
  		\centering
  		\includegraphics[scale=0.3]{reportImages/exp3_phase000001.PNG}
  		\caption{0.001\% of NxM = 3}
  	\end{subfigure}
  	
  	\caption{Varying the quantization levels for the magnitude}
  	\label{exp3_phase_1}
  \end{figure}
  As a conclusion for this part is has to be said that is much suitable to quantize in a lower number of levels the phase than the magnitude. In the case of reducing the number of levels in the magnitude the edges will be progressively affected and as a consequence the image will become blurry. In the case of reducing the number of levels for the phase, the image would like degraded remaining the edges.
  
\begin{thebibliography}{99}
%%

\bibitem{c1}
'Advanced Image Analysis Notes', Mathini Sellathurai, Heriot Watt University, 2014
%%
%%\bibitem{c3}
%%'Ultrasound Image Enhancement Based on Image Compounding' - Yair Kerner - Technion (Israel Institute of Technology) - Haifa - June 2004 
%%
%%\bibitem{c4}
%%'Physical Principles of General and Vascular Sonography' - Jim Baun - San Francisco, CA -March 2009
%%
%%\bibitem{c5}
%%'Physical Principles of General and Vascular Sonography' - Jim Baun - San Francisco, CA -March 2009
%%
%%\bibitem{c6}
%%'Image Formation and Image Processing in Ultrasound' - Jeffrey C. Bamber - Institute of Cancer Research and The Royal Marsden NHS Trust, Surrey, SM2 5PT
\end{thebibliography}

\end{document}
